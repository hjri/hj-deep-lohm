\documentclass[a4paper,12pt]{report} %размер бумаги устанавливаем А4, шрифт 12пунктов
\usepackage[utf8]{inputenc}
\usepackage[T2A]{fontenc}
\usepackage{geometry}
\usepackage{fancyhdr}
\usepackage{amsmath ,amsthm ,amssymb}
\usepackage{graphicx}
\usepackage{hyperref}
\usepackage{lipsum}

\renewcommand{\baselinestretch}{1.5} 

\usepackage{geometry} % Меняем поля страницы
\geometry{left=3cm}% левое поле
\geometry{right=2cm}% правое поле
\geometry{top=2cm}% верхнее поле
\geometry{bottom=2cm}% нижнее поле
\setlength{\parindent}{1.25cm}
\begin{document}
\begin{titlepage}
  \newpage

  \begin{center}
    ФЕДЕРАЛЬНОЕ АГЕНТСТВО ПО ОБРАЗОВАНИЮ РФ \\
    \vspace{1cm}
    Н-СКИЙ АРБУЗО-ЛИТЕЙНЫЙ ИНСТИТУТ \\*
    (ГОСУДАРСТВЕННЫЙ УНИВЕРСИТЕТ) \\*
    \hrulefill
  \end{center}

  \flushright{КАФЕДРА ХХХ}

  \vspace{8em}

  \begin{center}
    \Large Пояснительная записка \\ к дипломному проекту на тему:
  \end{center}

  \vspace{2.5em}

  \begin{center}
    \textsc{\textbf{исследование торсионных наногенераторов \linebreak стволовых клеток для борьбы с терроризмом}}
  \end{center}

  \vspace{6em}

  \begin{flushleft}
    Студент--дипломник \hrulefill Пупкин А.А. \\
    \vspace{1.5em}
    Научный руководитель \\
    доцент \hrulefill Иванов Б.Б.\\
    \vspace{1.5em}
    Рецензент \\
    к.ф.-м.н., в.н.с. АБВГ \hrulefill Петров В.В.\\
    \vspace{1.5em}
    Зав. кафедрой ХХХ \\
    д.ф-м.н, профессор \hrulefill Сидоров Г.Г.
  \end{flushleft}

  \vspace{\fill}

  \begin{center}
    Н-ск 2000
  \end{center}

\end{titlepage}% это титульный лист
\tableofcontents

\setlength{\parskip}{1em}
\section{Введение} %1 pages

Каждый день в мире взлетает и садится около 50 тысяч самолетов, среди
которых 30 тысяч - пасажирские, в воздухе одновременно их находятся около 10
тысяч. Кажется что для земного шара число достаточно маленькое, но если
посмотреть состояние в данный момент становится ясно что возушный траффик
сконтентирован в ограниченных областях.

Большинство самолетов - авиалинии - внутренние и внешние рейсы, следующие
примерно одними и теми же обозначенными маршрутами, и можно было бы просто
автоматизировать движение как железную дорогу, но помимо авиалиний в воздушном
пространстве так же могут быть и чартерные рейсы и военная авиация и метеозонды,
все это вносит большую долю неопределенности и непредсказуемости, особенно в
густонаселенных зонах - над одной лишь европой одновременно летят полторы тысячи
самолетов.

Возникает задача урегулирования воздушного траффика, которая на данный момент
решается тривиально - наземные станции наблюдения, связь диспетчер-пилот,
бортовые радары. Однако это все решения достаточно разрознены и требуют
формирования понимания ситуации на уровне человеческого восприятия или же некого
аппаратно-програмного обеспечения.

Наиболее прогрессивной является идея постороения сети на базе цифровой связи
самолет - самолет и самолет - наземная станция, в которой каждый узел может
передать и получить информацию о ближайшем возушном пространстве.
\newpage
\section{Современные проблемы авионики и автоматическое зависимое
  наблюдение-вещание} %8 pages

С развитием индустрии водушного транспорта - гражданская, космическая авиация,
самолето- и вертолётостроение - всё больше и больше возникает вопрос разработки
радиоэлектронной аппаратуры к воздушным судам. Требования к аппаратуре постоянно
растут - изначально от нее требовалось показывать пилотам информацию о положении
судна в воздухе, поддерживать курс и высоту (автопилот), обеспечивать связь с
диспетчером и экипажем, теперь же к требованиям добавляются диагностика лётных
агрегатов (двигатели, рулевые агрегаты, прочее), поддержание жизнеобеспечения
экипажа и пассажиров, радиолокация, информация о гео-положении
(GPS), внутренняя и внешняя видеорегистрация, и так далее. Технологический
прогресс ведет если не к полной автоматизации управлением суда, то как минимум к
его упрощению для человека.

Возникают две ключевые проблемы: внутренняя и внешняя.

Внутренняя заключается в том что с ростом требований к бортовой аппаратуре
становится все сложнее добавить новый функционал к старому ``железу'', затраты
на разработку программного обеспечения к ней так же растут. Проблема решается
разделением системы на модули, каждый из которых выполняет определенную для него
функцию, от получения данных до управления другими системами. Вместе все модули
формируют БРЭО - Бортовое Радиоэлектронное Оборудование. Разделение на модули
так же уменьшает стоимость разработки ПО - какие то модули можно купить готовыми
с уже написанным и рабочим ПО, использовать их в своей системе и разработать
только недостающие/желаемые модули

Внешняя проблема - проблема взаимодействия судов и наземных объектов. В воздухе
летает огромное разнообразие самолетов и вертолетов, самых разнообразных стран
производства и годов выпуска, разных назначений. Это приводит к тому что
какой-то борт может ``глушить'' все остальные из-за того что он использует
частоты какого-то старого стандарта или же это намеренное глушение какого-то
военного самолета. Некоторые самолеты современные и могут общаться друг с другом
по радиосвязи и обмениваться данными, некоторые могут, но более старому
протоколу, не совместимому с современным. У какого-то борта может быть
неисправен радар, и так далее.

Казалось бы - можно решить проблему совместимостей просто добавив еще
соотвествующих модулей, но, к сожалению, это приведет к излишней (возможно даже
``зашкаливающей'') сложности/запутанности БРЭО, а так же, вероятно, к лишнему
весу судна - например установка дополнительных антенн.

В данной работе я рассматриваю одну из функций БРЭО - автоматическое зависимое
наблюдение-вещание (АЗНВ), на английском - ADS-B (Automated dependent
survelliance-broadcast). 

АЗНВ - технология позволяющая пилотам и диспетчерам с высокой точностью получать
информацию о состоянии воздушного пространства - траффик, погодные условия,
аэронавигационная информация - в определенном радиусе ``вокруг себя''.
Абстрактно принцип работы - каждый борт вещает на определенной частоте иноформацию о себе и
слушает на определенной (возможно другой) частоте если передает ли какой другой
борт или наземная станция какую-то подобную информацию. Приемо-передатчик
передает эту информацию в другой модуль, который формирует ``картину мира''
вокруг себя и отрисовывает её на дисплее, возможно что отрисовкой занимается
другой модуль или ПО.

Система АЗНВ зависит от двух компонентов БРЭО - системы навигации (GPS, ГЛОНАСС)
и системы цифровой связи (data-link), в качестве последнего часто используют
модфицированный радар вторичного наблюдения (SSE, Secondary Survelliance Radar)
Mode S, работающий на частотах 978МГц или 1090МГц. В данной же работе
используется транспондер VDL Mode 4 (VHF Data-link, Очень высокочастная цифровая
связь режим 4, в дальшейшем - VDL-4).
\newpage
\section{А-Сеть и её реализация на основе стандарта VDL-4} %8 pages

Возможность цифровой связи между судами в воздухе и наземными станциями
наталкивает на идею формирования на их базе самоорганизующейся ad-hoc сети. В
такой сети каждый узел может получить информацию с любого другого узла, даже вне
радиуса действия своей радиоаппаратуры. Такая сеть не требует центрального узла
который бы поддерживал и контролировал сеть или её участок. Реализация этой
концепции - А-Сеть на основе стандарта VDL-4.

А-Сеть, очевидно, требует установленного на объекте транспондера VDL-4, но
помимо него требуется так же дополнительное оборудование, реализующее следующие
функции:
\begin{itemize}
\item измерение времени распространения сигнала от источника;
\item поддержка протоколов взаимодействия между объектами;
\item хранение текущего образа сети и его обновление;
\item формирование, хранение, доставку, поиск объекта назначения, маршрутизацию,
 коммутацию, контроль факта доставки и целостности пакета, включая пакеты
 передачи речи в цифровой форме;
\item вычисление расстояния между объектами по их координатным данным и времени
  распространения сигнала;
\end{itemize}

Каждый из объектов должен иметь свой уникальный адрес в сети, например можно
использовать уникальный номер в сети ATN.

Наземные и надводные неподвижные объекты с целью повышения точности определения
координат, корректирующего оборудования и контроля корректности в условиях
неуверенной работы глобальных систем спутниковой навигации (ГССН) (например, при
возбужденном состоянии ионосферы) могут быть привязаны:
\begin{itemize}
\item к географическим координатам (географическая привязка);
\item к задающему генератору Сети и/или UTC SU (временная привязка);
\item к сети ATN (сетевая привязка).
\item Последний вид привязки может быть использован для расширения зоны доступа
Сети к аэродромам, радиолокационным станциям, диспетчерским службам, станциям
слежения и т.п.
\end{itemize}



\section{Разработка функциональной схемы транспондера и технических требований к
  системе индикации} %8 pages
\lipsum [6]
\section{Разработка прикладного алгоритмического и
 програмного обеспения  системы индикации} %8 pages
\lipsum [7]
\section{Разработка вопросов по экологии и безопасности жизнидеятельности при
  работе с электронной аппаратурой} %8 pages
\lipsum [1]
\section{Технико-экономическеой обоснование. Сравнение системы индиации с
  аналогами с помощью метода анализа иерархий} %8 pages
\lipsum [1]
\section{Заключение}
\end{document}