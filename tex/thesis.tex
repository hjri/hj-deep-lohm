\documentclass[a4paper,12pt]{report} %размер бумаги устанавливаем А4, шрифт 12пунктов
\usepackage[utf8]{inputenc}
\usepackage[T2A]{fontenc}
\usepackage{geometry}
\usepackage{fancyhdr}
\usepackage{amsmath ,amsthm ,amssymb}
\usepackage{graphicx}
\usepackage{hyperref}
\usepackage{lipsum}

\renewcommand{\baselinestretch}{1.5} 

\usepackage{geometry} % Меняем поля страницы
\geometry{left=3cm}% левое поле
\geometry{right=2cm}% правое поле
\geometry{top=2cm}% верхнее поле
\geometry{bottom=2cm}% нижнее поле
\setlength{\parindent}{1.25cm}
\begin{document}
\begin{titlepage}
  \newpage

  \begin{center}
    ФЕДЕРАЛЬНОЕ АГЕНТСТВО ПО ОБРАЗОВАНИЮ РФ \\
    \vspace{1cm}
    Н-СКИЙ АРБУЗО-ЛИТЕЙНЫЙ ИНСТИТУТ \\*
    (ГОСУДАРСТВЕННЫЙ УНИВЕРСИТЕТ) \\*
    \hrulefill
  \end{center}

  \flushright{КАФЕДРА ХХХ}

  \vspace{8em}

  \begin{center}
    \Large Пояснительная записка \\ к дипломному проекту на тему:
  \end{center}

  \vspace{2.5em}

  \begin{center}
    \textsc{\textbf{исследование торсионных наногенераторов \linebreak стволовых клеток для борьбы с терроризмом}}
  \end{center}

  \vspace{6em}

  \begin{flushleft}
    Студент--дипломник \hrulefill Пупкин А.А. \\
    \vspace{1.5em}
    Научный руководитель \\
    доцент \hrulefill Иванов Б.Б.\\
    \vspace{1.5em}
    Рецензент \\
    к.ф.-м.н., в.н.с. АБВГ \hrulefill Петров В.В.\\
    \vspace{1.5em}
    Зав. кафедрой ХХХ \\
    д.ф-м.н, профессор \hrulefill Сидоров Г.Г.
  \end{flushleft}

  \vspace{\fill}

  \begin{center}
    Н-ск 2000
  \end{center}

\end{titlepage}% это титульный лист
\tableofcontents

\setlength{\parskip}{1em}
\section{Введение} %1 pages

Каждый день в мире взлетает и садится около 50 тысяч самолетов, среди
которых 30 тысяч - пасажирские, в воздухе одновременно их находятся около 10
тысяч. Кажется что для земного шара число достаточно маленькое, но если
посмотреть состояние в данный момент становится ясно что возушный траффик
сконтентирован в ограниченных областях, и поначалу может показаться что
урегулировать такую толпу невозможно.

Большинство самолетов - авиалинии - внутренние и внешние рейсы, следующие
примерно одними и теми же обозначенными маршрутами, и можно было бы просто
автоматизировать движение как железную дорогу, но помимо авиалиний в воздушном
пространстве так же могут быть и чартерные рейсы и военная авиация и метеозонды,
все это вносит большую долю неопределенности и непредсказуемости, особенно в
густонаселенных зонах - над одной лишь европой одновременно летят полторы тысячи
самолетов.

Возникает задача урегулирования воздушного траффика, которая на данный момент
решается тривиально - наземные станции наблюдения, связь диспетчер-пилот,
бортовые радары. Однако это все решения достаточно разрознены и требуют
формирования понимания ситуации на уровне человеческого восприятия или же некого
аппаратно-програмного обеспечения.

Наиболее прогрессивной является идея постороения сети на базе цифровой связи
самолет - самолет и самолет - наземная станция, в которой каждый узел может
передать и получить информацию о ближайшем возушном пространстве.
\newpage
\section{Современные проблемы авионики и автоматическое зависимое
  наблюдение-вещание} %8 pages

\lipsum [2-3]
\section{А-Сеть и её реализация на основе стандарта VDL-4} %8 pages
\lipsum [4-5]
\section{Разработка функциональной схемы транспондера и технических требований к
  системе индикации} %8 pages
\lipsum [6]
\section{Разработка прикладного алгоритмического и
 програмного обеспения  системы индикации} %8 pages
\lipsum [7]
\section{Разработка вопросов по экологии и безопасности жизнидеятельности при
  работе с электронной аппаратурой} %8 pages
\lipsum [1]
\section{Технико-экономическеой обоснование. Сравнение системы индиации с
  аналогами с помощью метода анализа иерархий} %8 pages
\lipsum [1]
\section{Заключение}
\end{document}