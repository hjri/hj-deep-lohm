\documentclass[a4paper,14pt]{report} %размер бумаги устанавливаем А4, шрифт 12пунктов
\usepackage[utf8]{inputenc}
\usepackage[T2A]{fontenc}
\usepackage{geometry}
\usepackage{fancyhdr}
\usepackage{amsmath ,amsthm ,amssymb}
\usepackage{graphicx}
\usepackage{hyperref}
\usepackage{lipsum}

\usepackage{geometry} % Меняем поля страницы
\geometry{left=3cm}% левое поле
\geometry{right=2cm}% правое поле
\geometry{top=2cm}% верхнее поле
\geometry{bottom=2cm}% нижнее поле

\begin{document}
\begin{titlepage}
  \newpage

  \begin{center}
    ФЕДЕРАЛЬНОЕ АГЕНТСТВО ПО ОБРАЗОВАНИЮ РФ \\
    \vspace{1cm}
    Федеральное государственное образовательное бюджетное учреждение высшего
    профессионального образования\\
    \textbf{Московский технический университет связи и информатики}
  \end{center}

  \vspace{1.5em}

  \hspace{-1cm}\begin{minipage}{0.35\textwidth}
    \begin{center}
      Разрешаю допустить к защите \\
      Зав. кафедрой \\
    \end{center}
    \vspace{-1.0em}
    \rule{\textwidth}{.1pt} \\
    <<\rule{2em}{.1pt}>> \rule{6em}{.1pt} 2016 г.
  \end{minipage}
  \vfill

  \begin{center}
    \Huge\bfseries Дипломный проект \\ на тему
  \end{center}

  \vspace{0.5em}

  \begin{center}
    \textsc{\textbf{\Large Разработка системы индикации для транспондера VLD-4}}
  \end{center}

  \vfill

  \begin{flushleft}
    Студент--дипломник \hrulefill / Костецкий Р.И. / \\
    \vspace{1.5em}
    Руководитель \hrulefill / Шаврин С.С. /\\
    \vspace{1.5em}
    Рецензент \hrulefill  / ????? ?.?. /\\
    \vspace{1.5em}
    Консультант по ЭБЖ \hrulefill / Курбатов В.А. /\\
    \vspace{1.5em}
    Консультант по ТЭО \hrulefill / Сиднев С.А. /\\
    \vspace{1.5em}
  \end{flushleft}

  \vspace{\fill}

  \begin{center}
    Москва 2016
  \end{center}

\end{titlepage}% это титульный лист
\tableofcontents
This is some  preamble  text  that  you  enter  yourself.
\section{Введение} %1 pages
Каждый день в мире взлетает и садится около 50 тысяч самолетов, среди
которых 30 тысяч - пасажирские, в воздухе одновременно находятся около 10 тысяч,
над одной лишь европой и европейской части Российской Федерации одновременно
летят полторы тысячи самолетов. Возникает проблема урегулирования воздушного
траффика, которая частично решается наземными станциями наблюдения и бортовыми
радарами, но отсутcвует некая глобальная система, позволяющая отобразить
общую картину состояния воздушного пространства.
\section{Современные проблемы авионики и автоматическое зависимое
  наблюдение-вещание} %8 pages
\lipsum [2-3]
\section{А-Сеть и её реализация на основе стандарта VDL-4} %8 pages
\lipsum [4-5]
\section{Разработка функциональной схемы транспондера и технических требований к
  системе индикации} %8 pages
\lipsum [6]
\section{Разработка прикладного алгоритмического и програмного обеспения
  системы индикации} %8 pages
\lipsum [7]
\section{Разработка вопросов по экологии и безопасности жизнидеятельности при
  работе с электронной аппаратурой} %8 pages
\lipsum [1]
\section{Технико-экономическеой обоснование. Сравнение системы индиации с
  аналогами с помощью метода анализа иерархий} %8 pages
\lipsum [1]
\section{Заключение}
\end{document}