\documentclass[a4paper,12pt]{report} %размер бумаги устанавливаем А4, шрифт 12пунктов
\usepackage[utf8]{inputenc}
\usepackage[T2A]{fontenc}
\usepackage{geometry}
\usepackage{fancyhdr}
\usepackage{amsmath ,amsthm ,amssymb}
\usepackage{graphicx}
\usepackage{gensymb}
\usepackage{hyperref}
\usepackage{lipsum}
\usepackage{graphicx}
\usepackage{titlesec}
\usepackage{listingsutf8}
\usepackage{hyperref}
\usepackage{amsmath}
\usepackage{hyphenat}
\usepackage{mathtools}
\renewcommand{\baselinestretch}{1.5} 
\usepackage[russian]{babel}
\setcounter{secnumdepth}{5}

\titleformat{\chapter}[block]
{\Large\bfseries}
{\thechapter.}{0.5em}{}

\titleformat{\section}[block]
{\large\bfseries}
{\thesection.}{0.5em}{}

\titleformat{\subsection}[block]
{\large}
{\thesubsection.}{0.5em}{}

\usepackage{color}

\definecolor{lightgray}{rgb}{.9,.9,.9}
\definecolor{darkgray}{rgb}{.4,.4,.4}
\definecolor{purple}{rgb}{0.65, 0.12, 0.82}

\lstdefinelanguage{JavaScript}{
  keywords={typeof, new, true, false, catch, function, return, null, catch, switch, var, if, in, while, do, else, case, break},
  keywordstyle=\color{blue}\bfseries,
  ndkeywords={class, export, boolean, throw, implements, import, this},
  ndkeywordstyle=\color{darkgray}\bfseries,
  identifierstyle=\color{black},
  sensitive=false,
  comment=[l]{//},
  morecomment=[s]{/*}{*/},
  commentstyle=\color{purple}\ttfamily,
  stringstyle=\color{red}\ttfamily,
  morestring=[b]',
  morestring=[b]"
}

\lstset{
  language=JavaScript,
  backgroundcolor=\color{lightgray},
  extendedchars=true,
  basicstyle=\footnotesize\ttfamily,
  showstringspaces=false,
  showspaces=false,
  numbers=left,
  numberstyle=\footnotesize,
  numbersep=9pt,
  tabsize=2,
  breaklines=true,
  showtabs=false,
  captionpos=b
}

\usepackage{geometry} % Меняем поля страницы
\geometry{left=3cm}% левое поле
\geometry{right=2cm}% правое поле
\geometry{top=2cm}% верхнее поле
\geometry{bottom=2cm}% нижнее поле
\setlength{\parindent}{1.25cm}

\begin{document}
\begin{titlepage}
  \newpage

  \begin{center}
    ФЕДЕРАЛЬНОЕ АГЕНТСТВО ПО ОБРАЗОВАНИЮ РФ \\
    \vspace{1cm}
    Федеральное государственное образовательное бюджетное учреждение высшего
    профессионального образования\\
    \textbf{Московский технический университет связи и информатики}
  \end{center}

  \vspace{1.5em}

  \hspace{-1cm}\begin{minipage}{0.35\textwidth}
    \begin{center}
      Разрешаю допустить к защите \\
      Зав. кафедрой \\
    \end{center}
    \vspace{-1.0em}
    \rule{\textwidth}{.1pt} \\
    <<\rule{2em}{.1pt}>> \rule{6em}{.1pt} 2016 г.
  \end{minipage}
  \vfill

  \begin{center}
    \Huge\bfseries Дипломный проект \\ на тему
  \end{center}

  \vspace{0.5em}

  \begin{center}
    \textsc{\textbf{\Large Разработка системы индикации для транспондера VLD-4}}
  \end{center}

  \vfill

  \begin{flushleft}
    Студент--дипломник \hrulefill / Костецкий Р.И. / \\
    \vspace{1.5em}
    Руководитель \hrulefill / Шаврин С.С. /\\
    \vspace{1.5em}
    Рецензент \hrulefill  / ????? ?.?. /\\
    \vspace{1.5em}
    Консультант по ЭБЖ \hrulefill / Курбатов В.А. /\\
    \vspace{1.5em}
    Консультант по ТЭО \hrulefill / Сиднев С.А. /\\
    \vspace{1.5em}
  \end{flushleft}

  \vspace{\fill}

  \begin{center}
    Москва 2016
  \end{center}

\end{titlepage}% это титульный лист
\tableofcontents

\setlength{\parskip}{1em}
\chapter*{Введение} %1 pages
\addcontentsline{toc}{chapter}{Введение}

Каждый день в мире взлетает и садится около 50 тысяч самолетов, среди
которых 30 тысяч -- пассажирские, в воздухе одновременно их находятся около 10
тысяч. Кажется что для земного шара число достаточно маленькое, но если
посмотреть состояние в данный момент становится ясно что воздушный трафик
сконденсирован в ограниченных областях.

Большинство самолетов -- авиалинии: внутренние и внешние рейсы, следующие
примерно одними и теми же обозначенными маршрутами, и можно было бы просто
автоматизировать движение как железную дорогу, но помимо авиалиний в воздушном
пространстве так же могут быть и чартерные рейсы и военная авиация и метеозонды,
все это вносит большую долю неопределенности и непредсказуемости, особенно в
густонаселенных зонах -- над одной лишь Европой одновременно летят полторы тысячи
самолетов.

Возникает задача урегулирования воздушного трафика, которая на данный момент
решается тривиально: наземные станции наблюдения, связь диспетчер--пилот,
бортовые радары. Однако это все решения достаточно разрознены и требуют
формирования понимания ситуации на уровне человеческого восприятия или же некого
аппаратно--программного обеспечения.

Наиболее прогрессивной является идея построения сети на базе цифровой связи
самолет--самолет и самолет--наземная станция, в которой каждый узел может
передать и получить информацию о ближайшем воздушном пространстве.
\newpage
\chapter{Современные проблемы авионики и автоматическое зависимое
  наблюдение--вещание} %8 pages

С развитием индустрии воздушного транспорта - гражданская, космическая авиация,
самолето-- и вертолётостроение - всё больше и больше возникает вопрос разработки
радиоэлектронной аппаратуры к воздушным судам. Требования к аппаратуре постоянно
растут - изначально от нее требовалось показывать пилотам информацию о положении
судна в воздухе, поддерживать курс и высоту (автопилот), обеспечивать связь с
диспетчером и экипажем, теперь же к требованиям добавляются диагностика лётных
агрегатов (двигатели, рулевые агрегаты, прочее), поддержание жизнеобеспечения
экипажа и пассажиров, радиолокация, информация о гео--положении
(GPS), внутренняя и внешняя видеорегистрация, и так далее. Технологический
прогресс ведет если не к полной автоматизации управлением суда, то как минимум к
его упрощению для человека.

\section{Основные проблемы}

\subsection{Внутренняя}

Внутренняя заключается в том что с ростом требований к бортовой аппаратуре
становится все сложнее добавить новый функционал к старому ``железу'', затраты
на разработку программного обеспечения к ней так же растут. Проблема решается
разделением системы на модули, каждый из которых выполняет определенную для него
функцию, от получения данных до управления другими системами. Вместе все модули
формируют БРЭО - Бортовое Радиоэлектронное Оборудование. Разделение на модули
так же уменьшает стоимость разработки ПО - какие то модули можно купить готовыми
с уже написанным и рабочим ПО, использовать их в своей системе и разработать
только недостающие/желаемые модули

\subsection{Внешняя}

Внешняя проблема - проблема взаимодействия судов и наземных объектов. В воздухе
летает огромное разнообразие самолетов и вертолетов, самых разнообразных стран
производства и годов выпуска, разных назначений. Это приводит к тому что
какой--то борт может ``глушить'' все остальные из--за того что он использует
частоты какого--то старого стандарта или же это намеренное глушение какого--то
военного самолета. Некоторые самолеты современные и могут общаться друг с другом
по радиосвязи и обмениваться данными, некоторые могут, но более старому
протоколу, не совместимому с современным. У какого--то борта может быть
неисправен радар, и так далее.

Казалось бы - можно решить проблему совместимости просто добавив еще
соответствующих модулей, но, к сожалению, это приведет к излишней (возможно даже
``зашкаливающей'') сложности/запутанности БРЭО, а так же, вероятно, к лишнему
весу судна - например установка дополнительных антенн.

В данной работе я рассматриваю одну из функций БРЭО - автоматическое зависимое
наблюдение--вещание (АЗНВ), на английском - ADS--B (Automated dependent
survelliance--broadcast).

\section{Что такое АЗНВ}

АЗНВ - технология позволяющая пилотам и диспетчерам с высокой точностью получать
информацию о состоянии воздушного пространства - трафик, погодные условия,
аэронавигационная информация - в определенном радиусе ``вокруг себя''.
Абстрактно принцип работы - каждый борт вещает на определенной частоте информацию о себе и
слушает на определенной (возможно другой) частоте если передает ли какой другой
борт или наземная станция какую--то подобную информацию. Приемо--передатчик
передает эту информацию в другой модуль, который формирует ``картину мира''
вокруг себя и отрисовывает её на дисплее, возможно что отрисовкой занимается
другой модуль или ПО.

Система АЗНВ зависит от двух компонентов БРЭО - системы навигации (GPS, ГЛОНАСС)
и системы цифровой связи (data--link), в качестве последнего часто используют
модифицированный радар вторичного наблюдения (SSE, Secondary Survelliance Radar)
Mode S, работающий на частотах 978МГц или 1090МГц. В данной же работе
используется транспондер VDL Mode 4 (VHF Data--link, Очень высокочастотная цифровая
связь; режим 4, в дальшейшем - VDL--4).
\newpage

\chapter{А--Сеть и её реализация на основе стандарта \\
 VDL--4} %8 pages

\section{Концепция А--Сети}

Возможность цифровой связи между судами в воздухе и наземными станциями
наталкивает на идею формирования на их базе самоорганизующейся ad--hoc сети. В
такой сети каждый узел может получить информацию с любого другого узла, даже вне
радиуса действия своей радиоаппаратуры. Такая сеть не требует центрального узла
который бы поддерживал и контролировал сеть или её участок. Реализация этой
концепции - А--Сеть на основе стандарта VDL--4. \cite{theconcept}

\section{Требования А--Сети}

А--Сеть, очевидно, требует установленного на объекте транспондера VDL--4, но
помимо него требуется так же дополнительное оборудование, реализующее следующие
функции:
\begin{itemize}
\item измерение времени распространения сигнала от источника;
\item поддержка протоколов взаимодействия между объектами;
\item хранение текущего образа сети и его обновление;
\item формирование, хранение, доставку, поиск объекта назначения, маршрутизацию,
  коммутацию, контроль факта доставки и целостности пакета, включая пакеты
  передачи речи в цифровой форме;
\item вычисление расстояния между объектами по их координатным данным и времени
  распространения сигнала;
\end{itemize}

Каждый из объектов должен иметь свой уникальный адрес в сети, например можно
использовать уникальный номер в сети ATN.

Наземные и надводные неподвижные объекты с целью повышения точности определения
координат, корректирующего оборудования и контроля корректности в условиях
неуверенной работы глобальных систем спутниковой навигации (ГССН) (например, при
возбужденном состоянии ионосферы) могут быть привязаны:
\begin{itemize}
\item к географическим координатам (географическая привязка);
\item к задающему генератору Сети и/или UTC SU (временная привязка);
\item к сети ATN (сетевая привязка).
\end{itemize}
Последний вид привязки может быть использован для расширения зоны доступа
Сети к аэродромам, радиолокационным станциям, диспетчерским службам, станциям
слежения и т.п.

Географически привязанные объекты должны устанавливаться на отдельных
геодезических постаментах для снижения погрешностей, связанных с сезонными
колебаниями грунта. Конструкция их антенн должна обеспечивать необходимую
защищенность от эффектов многолучевого распространения и затенения со стороны
препятствий.

Сетевые адреса (номера) объектов, имеющих географическую, временную и сетевую
привязки, должны быть известны всем объектам сети, помимо этого рекомендуется
использовать системы глобальной навигации разных стандартов для повышения
точности.

\section{Возможности предоставляемые А--Сетью}

\begin{itemize}
\item Точность - При наличии в сети четырех и более объектов возможно
  определение реальных координат их всех, а так же и внешних объектов. 
\item Маштабируемость - возможность расширения зоны наблюдения в пределах
  доступа сети.
\item Надежность - возможность маршрутизации ``в обход'' позволяет сохранить
  соединение при возникновении помехи. Разлом в сети не выводит её из строя -
  каждый фрагмент сохраняет способность автономного функционирования.
\item Защищенность - наличие расчета расстояния до объекта (и последующее
  вычисление реальных координат) позволяет исключать из образа сети ложные
  объекты.
\item Поиск по сети - при необходимости передать сообщение конкретному объекту
  может быть использован принцип ``штурма'' - широковещательная рассылка запроса
  на соединение с защитой от повторной передачи по одному и тому же участку
\item Приоритезация - Наличие образа в Сети на объекте дает возможность
  обеспечить ``ручную'' маршрутизацию сообщения или автоматическую в
  соответствие с его рангом. Срочное сообщение следует маршрутизировать с
  позиции минимального количества переприемов,  вносящих основную задержку.
  Сообщение, требующее повышенной степени достоверности передачи, следует
  маршрутизировать через максимально короткие радиоканалы, обладающие (в
  статистике) лучшей помехоустойчивостью. Маршруты сообщений, не требующих
  специального внимания, должны минимизировать относительные потери пропускной
  способности на ребрах сети. 
\item Голосовая связь - в случае экстренной ситуации или при невозможности
  прямого радиодоступа есть возможность установить речевую связь.
\end{itemize}
\newpage

\chapter{Разработка функциональной схемы транспондера и технических требований к
  системе индикации} %8 pages 

\section{Трансподнер VDL--4}

\begin{figure}[!ht]
  \includegraphics[width=0.9\textwidth]{vdl4}
  \caption{функциональная схема транспондера VDL--4}
\end{figure}

\begin{itemize}
\item [ЦПОС] Центральный процессор обработки сигналов
\item [МШУ] Малошумящий усилитель
\item [ГУН] Генератор управляемый напряжением
\item [ПФ] Полосовой фильтр
\item [АЦП] Аналогово--цифровой преобразователь
\item [ЦАП] Цифро--аналоговый преобразователь
\end{itemize}
\newpage

\section{Система индикации}

Система индикации будет получать данные от транспондера в виде сформированного
текстового файла, возможна реализация тестового режима в котором будут показаны
тестовые (не соответствующие действительности) объекты сети, с целью проверки
работоспособности системы индикации как таковой.

\subsection{Основные требования}

Система индикации должна:
\begin{itemize}
\item Показывать текущее положение объекта на котором установлена данная система
  (себя)
\item Показывать радиус действия прямого радиодоступа - 400км
\item Показывать другие объекты сети:
  \begin{itemize}
  \item Самолеты, находящиеся в радиусе прямого радиодоступа
  \item Самолеты, находящиеся в зоне действия сети (при запросе расширения зоны
    наблюдения)
  \item Наземные и водные станции
  \end{itemize}
\item Планируемый курс определенного объекта (по запросу)
\end{itemize}

Для тестового режима, реализуемого в данной работе реализуем следующий
функционал:
\begin{itemize}
\item Показываем себя - самолет летящий с фиксированным курсом и скоростью из
  начальных координат г. Москвы.
\item 2--3 самолета, летящих недалеко от нас с фиксированными курсами и
  скоростями.
\item Самолеты, покинувшие область прямого радиодоступа будут помечаться как
  самолеты ``в сети'' пока расстояние до них не станет больше 800км, после чего
  они полностью исчезнут с карты, до тех пор пока мы не встретимся с ними снова.
\end{itemize}

\subsection{Технические требования}

Система индикации должна работать на основных платформах: Microsoft\copyright
Windows\copyright, Mac OS X, Linux с системой окон X11. Так же желательна
поддержка мобильных систем Android и Apple iOS.
\newpage

\chapter{Разработка прикладного алгоритмического и  программного
  обеспечения  системы индикации} %8 pages 
\section{Выбор платформы}

Для поддержания наибольшего числа платформ и легкости разработки, поддержки и
разработки системы индикации, я решил использовать платформу HTML5. Система
индикации (в дальшейшем СИ) написана на языке JavaScript с использованием
нескольких сторонних библиотек.

\subsection{Особенности платформы}

СИ может быть запущена из практически любого современного браузера - IE11,
Firefox, Chrome, Safari, но в целях удобства и потенциальной возможности доступа
к аппаратной части я использовал оболочку Electron\cite{electron} - окружение позволяющее
написать приложение целиком на JavaScript\\HTML5 и при этом иметь функционал
``родного'' приложения - доступ к файловой системе, внешним устройствам,
функционалу конкретной операционной системе (уведомления, интеграция).

\subsection{Используемые технологии}

СИ в большей части использует библиотеку обработки и отображения данных - d3.js \cite{d3js}.
Данная библиотека позволяет, на основании входных данных выполнять определенные
действия - создавать, изменять, удалять указанные узлы. В данной работе при
помощи этой библиотеки управляется SVG элементы, строящие векторное
представление ортографической проекции земного шара (границы суши и земли) и
объектов находящихся на нём. Библиотека так же позволяет напрямую использовать
координаты в формате широта--долгота.

\subsection{Алгоритм программы}

Алгоритм программы заключается в следующем:
\begin{itemize}
\item Получить данные от транспондера (или сформировать их для тестового режима)
\item Нормализовать данные, обновить внутренний образ сети на их основе (в
  тестовом режиме данные обновляются за счет предыдущего состояния)
\item Передать данные о координатах в объекты d3.js
\item Модифицировать объекты d3.js для отображения дополнительной информации
  (направление, бортовой номер, скорость, тип)
\item Повторить все шаги снова по прохождении интервала обновления
\end{itemize}
\newpage

Исходный код программы:

\lstinputlisting[caption=client/index.js]{../client/index.js}
\lstinputlisting[caption=app.js]{../app.js}
\lstinputlisting[caption=index.html]{../index.html}

\newpage
\chapter{Разработка инструкции по эксплуатации}
\newpage

\chapter{Разработка вопросов по экологии и безопасности жизнидеятельности при
  работе с электронной аппаратурой (ЭА)} %8 pages
\section{Эргономические требования к рабочему месту при работе с ЭА}

Основные требования к рабочему месту оператора наземной станции:

\begin{itemize}
\item правильное размещение рабочего места в производственном помещении; 
\item выбор эргономично обоснованного рабочего положения, производственных
  мебели с учетом антропометрическими характеристик человека; 
\item рациональное расположение оборудования. 
\end{itemize}

Общие принципы организации рабочего места:

\begin{itemize}
\item на рабочем месте не должно быть ничего лишнего. Все Необходимые для работы
  предметы должны быть рядом с работником, но не мешать ему; 
\item то предметы, которыми пользуются чаще, располагаются ближе, чем предметы,
  которыми пользуются реже; 
\item если используют обе руки, то местоположение приспособлений выбирается с
  учетом удобства захвата его двумя руками; 
\item рабочее место не должно быть загромождено; 
\item организация рабочего места должна обеспечивать необходимую обзорность 
\end{itemize}

Статические напряжения работника в процессе труда связаны с поддержания в
неподвижно состоянии предметов и орудий труда, а также поддержание рабочей позы 
% нормы
\section{Микроклиматические условия в производственном помещении  при работе с
  ЭА} 
% температура, влажность, нормы
Нормы производственного микроклимата установлены в СанПиН 2.2.4.548-96
«Гигиенические требования к микроклимату производственных помещений» 

Они едины для всех производств и всех климатических зон с некоторыми
незначительными отступлениями. 

В этих нормах отдельно нормируется каждый компонент микроклимата в рабочей зоне
производственного помещения: температура, относительная влажность, скорость
движения воздуха в зависимости от способности организма человека к
акклиматизации в разное время года, характера одежды, интенсивности производимой
работы и характера тепловыделений в рабочем помещении. 

Работа оператора наземной станции подходит под категорию Iа -- работы с
интенсивностью энерготрат до 120 ккал/ч (до 139 Вт), производимые сидя и
сопровождающиеся незначительным физическим напряжением (ряд профессий на 
предприятиях точного приборо-- и машиностроения, на часовом, швейном
производствах, в сфере управления и т.п.). 

\begin{flushleft}

\begin{table}[!h]
  \caption{Требования по микроклимату к категории Iа}
  \begin{tabular}{|l|p{2.75cm}|p{2.75cm}|p{2.75cm}|p{2.75cm}|}
    \hline
    Период года & Темп. воздуха, C\degree & \nohyphens{Темп. поверхностей, C\degree} & \nohyphens{Отн. влажность воздуха, \%} & \nohyphens{Скорость движения воздуха, \%}\\
    \hline
    Холодный    & 22---24               & 21---25                    & 60---40                           & 0.1                       \\
    \hline
    Теплый      & 23---25               & 22---26                    & 60---40                           & 0.1                     \\
    \hline
  \end{tabular}
\end{table}
\end{flushleft}

\section{Освещенность в производственном помещении при работе  с ЭА}

Рассчитаем необходимое количество осветительных приборов в производственном
помещении -- комнате операторов наземной станции. В соответствии с нормами СанПиН
2.2.2/2.4.1340-03 освещенность на поверхности стола должна быть 300--500лк,
возьмем среднее -- $E_{min} = 400лк$.

Необходимое количество приборов высчитывается по формуле:

\[
N = \frac{ E_{min} \cdot S \cdot k }{ F_{\Lambda} \cdot z \cdot n \cdot \eta }
\]

где \\
$S$ -- площадь пола в помещении \\
$k$ -- коэффициент запаса \\
$F_{\Lambda}$ -- световой поток, создаваемый одной лампой. В нашем случае
используем лампы ИКЕЯ ЛЕДАРЕ с заявленным световым потоком $400$ лм\cite{ledare} \\
$n$ -- количество ламп в светильнике -- в нашем случае -- $1$ \\
$\eta$ -- коэффициент использования светового потока, зависит от показателя
помещения $\varphi$, коэффициентов отражения  ($\rho_{\text{стен}}$) и потолка
($\rho_{\text{пот}}$) и рассчитывается из таблицы\\
$z$ -- коэффициент неравномерности освещенности, пример равным $0.8$

Показатель помещения $\varphi$ рассчитывается по формуле:

\[
\varphi = \frac{A \cdot B}{H_{P} \cdot (A + B)}
\]

Где $A$ и $B$ -- ширина и длина помещения, м; $H_{P}$ -- высота подвеса
светильника, расстояние между рабочей поверхностью и светильником. В нашем
случае возьмем $A = B = 3\text{м}$ а $H_{P} = 1.4\text{м}$. Показатель помещения
будет равен 

\[
\varphi = \frac{3^2}{1.4*(3*2)} = \frac{9}{1.4*6} = \frac{9}{8.4} = 1.07
\]

По таблице \cite{ledcoef} найдем значение $\eta$ приняв коэффициенты отражения
пола и потолка 70\% и 50\% соответственно.

$\eta = 49$

коэффициент запаса для светодиодных ламп - $k = 11$\cite{ledcoef2}

Рассчитаем необходимое количество осветительных приборов:

\[
N = \frac{ 400 \cdot 9 \cdot 1.1 }{ 400 \cdot 0.8 \cdot 1 \cdot 49 } = 0.25
\]

Для такого помещения достаточно одной ламы.
\section{Режим труда и отдыха при работе с ЭА}

Согласно СанПиН 3.3.2.007-98 в течение дня должны предусматриваться:

\begin{itemize}
\item перерывы для отдыха и употребления пищи (обеденные перерывы);
\item перерывы для отдыха и личных нужд (согласно трудовым нормам);
\item дополнительные перерывы, которые вводятся для отдельных профессий с учетом особенностей трудовой деятельности.
\end{itemize}

При работе оператор наземной станции должен выходить на перерыв минимум три
раза, из которых 30 мин обеденный перерыв и 30 минут на личные нужды и отдых.

\section{Вывод}

В данной работе мною были рассмотрены основные вопросы безопасности
жизнедеятельности и охраны труда оператора наземной станции наблюдения. Для того
чтобы поддерживать охрану труда на рабочем месте нужно придерживаться расчитаных
и выписанных мною норм.
\newpage
\chapter{Технико--экономические обоснование. Сравнение системы индиации с
  аналогами с помощью метода анализа иерархий} %8 pages 
\section{Постановка задачи}

Цель расчетов в данной задаче - сравнить использование данной системы индикации
с аналогами используя метод анализа иерархий (МАИ).

\subsection{Исходные данные}

Все приложения работают с любыми транспондерами VDL--4. Оборудование транспондера
не рассматривается, так же не рассматриваются комплекты оборудования + ПО.

\subsubsection{Система индикации рассматриваемая в данной работе (СИ)}

Разрабатываемая мной система может работать на практически любой платформе и на
большом количестве возможного оборудования, распространяется абсолютно бесплатно
по лицензии GNU GPL v3. Однако, к сожалению, кроме меня обеспечивать техническую
поддержку клиентов некому.

\subsubsection{ADB--B View}

Приложение для iPhone и iPad от компании FreeFlight. Распространяется бесплатно,
полный функционал доступен изначально (нет ``докупаемого'' функционала). Компания предоставляет полную техническую
поддержку, но исходный код недоступен и лицензия коммерческая.

\url{https://itunes.apple.com/us/app/ads-b-view/id563077606?mt=8}

\subsubsection{Avare}

Приложение для устройств с операционной системой Android. Распространяется так
же бесплатно с полным функционалом, под лицензией Apache 2. Техническая
поддержка предоставляется посредством форума сообщества.

\url{https://play.google.com/store/apps/details?id=com.ds.avare&hl=en}


\begin{table}[!h]
  \caption{Сравниваемые системы и показатели}
  \begin{tabular}{ | r | p{6cm} | l | l |}
    \hline
    Показатель   & СИ рассм. в данной работе                  & ADS--B View   & Avare  \\
    \hline
    Платформы    & Win, Mac, Linux, FreeBSD, Android, iOS, WP & iOS           & Android \\
    \hline
    Лицензия ПО  & GPLv3                                      & Коммерческая  & Apache 2.0 \\
    \hline
    Поддержка    & Авторская (по возможности)                 & Коммерческая  & Сообщество \\
    \hline
  \end{tabular}
\end{table}

\section{Иерархическое представление задачи}

На верхнем иерархическом уровне основная задача - выбрать наилучший вариант
системы индикации, в сравнении с несколькими аналогами. На промежуточных уровнях
- выбор альтернатив по системе критериев.

Нам необходимо определиться с важностью критериев при выборе наилучшего
варианта. Приоритет определяется путем попарного сравнения критериев.

\subsection{Сравнение критериев}

Сравним в деталях каждый показатель (критерий) с каждым.
\subsubsection{Платформы vs. Лицензия}

Количество поддерживаемых платформ дает больший выбор используемого ``железа''
для системы, не обязательно покупать оборудование конкретного производителя
из--за того что ПО этого требует.

С другой стороны открытая лицензия позволяет (как правило) иметь доступ к
исходному коду ПО и возможность его изменять - это дает возможность
самостоятельно исправить неполадку не дожидаясь официальной тех.поддержки и
бюрократии, однако в целом принято считать что коммерческие продукты обладают
лучшим качеством в сравнении с открытыми.

\textbf{Итого:} 2 к 1 в пользу платформ
\subsubsection{Лицензия vs. Поддержка}

Критерий лицензии обсуждался выше. Техническая поддержка ПО зачастую имеет
решающую роль при выборе программного решения. В нашем случае, кажется что
лицензия и поддержка равносильны и взаимоисключаемы - либо есть исходный код и
его можно править, либо есть техподдержка которая исправит любую неполадку. Оба
случая видятся равноценными по затратам.

\textbf{Итого:} 1 к 1
\subsubsection{Платформы vs. Поддержка}

Помимо сказанного выше стоит добавить пару слов конкретно к этому сравнению. ПО
написанное под несколько платформ сразу зачастую пишется на ЯП с низким порогом
вхождения (Java, JavaScript), что облегчает поддержку ПО собственными
средствами.

Так же хочется сказать что при неисправности платформы, если ПО зависит от неё
придется общаться с техподдержкой платформы, в то время как платформонезависимое
ПО можно просто установить на другой платформе.

\textbf{Итого:} 2 к 1 в пользу платформ

\subsubsection{Заключение}

Построим матрицу сравнений
\begin{table}[h]
  \caption{матрица сравнений}
  \begin{tabular}{l|c|c|c|}
    \cline{2-4}
    {}                              & \multicolumn{1}{l|}{Платформы} & \multicolumn{1}{l|}{Лицензия} & \multicolumn{1}{l|}{Поддержка} \\ \hline
    \multicolumn{1}{|l|}{Платформы} & 1                              & 2                             & 2                              \\ \hline
    \multicolumn{1}{|l|}{Лицензия}  & $0.5$                          & 1                             & 1                              \\ \hline
    \multicolumn{1}{|l|}{Поддержка} & $0.5$                          & 1                             & 1                              \\ \hline
  \end{tabular}
\end{table}

\subsection{Сравнение альтернатив}
Аналогично сравним альтернативы по критериям

\begin{table}[h]
  \caption{матрица сравнений по платформам}
  \begin{tabular}{l|c|c|c|}
    \cline{2-4}
    {}                               & \multicolumn{1}{l|}{СИ}        & \multicolumn{1}{l|}{ADS--B View} & \multicolumn{1}{l|}{Avare} \\ \hline
    \multicolumn{1}{|l|}{СИ}         & 1                              & 8                               & 6                          \\ \hline
    \multicolumn{1}{|l|}{ADS--B View} & 0.125                          & 1                               & 0.5                        \\ \hline
    \multicolumn{1}{|l|}{Avare}      & 0.165                          & 2                               & 1                          \\ \hline
  \end{tabular}
\end{table}

\begin{table}
  \caption{матрица сравнений по лицензиям}
  \begin{tabular}{l|c|c|c|}
    \cline{2-4}
    {}                               & \multicolumn{1}{l|}{СИ}        & \multicolumn{1}{l|}{ADS--B View} & \multicolumn{1}{l|}{Avare} \\ \hline
    \multicolumn{1}{|l|}{СИ}         & 1                              & 3                               & 1                          \\ \hline
    \multicolumn{1}{|l|}{ADS--B View} & 0.333                          & 1                               & 0.333                      \\ \hline
    \multicolumn{1}{|l|}{Avare}      & 1                              & 3                               & 1                          \\ \hline
  \end{tabular}
\end{table}

\begin{table}
  \caption{матрица сравнений по поддержке}
  \begin{tabular}{l|c|c|c|}
    \cline{2-4}
    {}                               & \multicolumn{1}{l|}{СИ}        & \multicolumn{1}{l|}{ADS--B View} & \multicolumn{1}{l|}{Avare} \\ \hline
    \multicolumn{1}{|l|}{СИ}         & 1                              & 0.165                           & 0.25                       \\ \hline
    \multicolumn{1}{|l|}{ADS--B View} & 6                              & 1                               & 2                          \\ \hline
    \multicolumn{1}{|l|}{Avare}      & 4                              & 0.5                             & 1                          \\ \hline
  \end{tabular}
\end{table}
\subsection{Синтез приоритетов}

Вычислим вектора приоритетов для критериев и альтернатив

\subsubsection{Критерии}

Для начала вычисляем среднее геометрическое для каждого критерия.
$ b_1=\sqrt[3]{1 \cdot 2 \cdot 2} = \sqrt[3]{2} = 1.2599210499 $ \\
$ b_2=\sqrt[3]{0.5 \cdot 1 \cdot 1} = \sqrt[3]{0.5} = 0.793700525984 $ \\
$ b_3=\sqrt[3]{0.5 \cdot 1 \cdot 1} = \sqrt[3]{0.5} = 0.793700525984 $ \\

Находим сумму расчитаных средний геометрических
\\
$ S = 1.2599210499 + 0.793700525984 + 0.793700525984 = 2.84732210 $ \\

И находим векторы приоритетов, разделив среднее геометрическое на сумму всех.
\\
$K_1 = b_1/S = 0.44249333 $ \\
$K_2 = b_2/S = 0.35120719 $ \\
$K_3 = b_3/S = 0.35120719 $ \\

\subsubsection{Платформы}

Повторяем те же действия но уже для каждой системы относительно критериев.
$ b_1=\sqrt[3]{1 \cdot 8 \cdot 6} = \sqrt[3]{48} = 3.63424118566 $ \\
$ b_2=\sqrt[3]{0.125 \cdot 1 \cdot 0.5} = \sqrt[3]{0.0625} = 0.396850262992 $ \\
$ b_3=\sqrt[3]{0.165 \cdot 2 \cdot 1} = \sqrt[3]{0.33} = 0.691042323001 $ \\
\\
$ S = 3.63424118566 + 0.396850262992 + 0.691042323001 = 4.72213377 $ \\
\\
$ X_1 = b_1/S = 0.76961843 $ \\
$ X_2 = b_2/S = 0.08404045 $ \\
$ X_3 = b_3/S = 0.14634112 $ \\

\subsubsection{Лицензия}

$ b_1=\sqrt[3]{1 \cdot 3 \cdot 1} = \sqrt[3]{3} = 1.44224957031 $ \\
$ b_2=\sqrt[3]{0.333 \cdot 1 \cdot 0.333} = \sqrt[3]{0.110889} = 0.480429303424
$ \\
$ b_3=\sqrt[3]{1 \cdot 3 \cdot 1} = \sqrt[3]{3} = 1.44224957031 $ \\
\\
$ S = 1.44224957031 + 0.480429303424 + 1.44224957031 = 3.36492844 $ \\
\\
$ Y_1 = b_1/S = 0.42861226 $ \\
$ Y_2 = b_2/S = 0.14277549 $ \\
$ Y_3 = b_3/S = 0.42861226 $ \\

\subsubsection{Поддержка}
$ b_1=\sqrt[3]{1 \cdot 0.165 \cdot 0.25} = \sqrt[3]{0.04125} = 0.345521161501 $ \\
$ b_2=\sqrt[3]{6 \cdot 1 \cdot 2} = \sqrt[3]{12} = 2.28942848511 $ \\
$ b_3=\sqrt[3]{4 \cdot 0.5 \cdot 1} = \sqrt[3]{2} = 1.2599210499 $ \\
\\
$ S = 0.345521161501 + 2.28942848511 + 1.2599210499 = 3.89487070 $ \\
\\
$ Z_1 = b_1/S = 0.08871184 $ \\
$ Z_2 = b_2/S = 0.58780603 $ \\
$ Z_3 = b_3/S = 0.32348212 $ \\

\subsection{Расчет глобальных приоритетов}

Рассчитав все векторы приоритетов найдем глобальные приоритеты, перемножив
векторы приоритетов альтернатив с соответствующими векторами критериев:\\

$P_i=X_i \cdot K_1 + Y_i \cdot K_2 + Z_i \cdot K_3 $ \\
\\
$P_{СИ} = 0.44249333 * 0.76961843 + 0.35120719 * 0.42861226 + 0.35120719 *
0.08871184 = 0.52223897$ \\
$P_{ADS-B View} = 0.44249333 * 0.08404045 + 0.35120719 * 0.14277549 + 0.35120719
* 0.58780603 = 0.29377282 $ \\
$P_{Avare} = 0.44249333 * 0.14634112 + 0.35120719 * 0.42861226 + 0.35120719 *
0.32348212 = 0.32889592$ \\
\\
\subsection{Вывод}

Из вычислений видно что наилучший выбор - СИ реализуемая в данной работе.
Наилучший по большему счету из--за количества поддерживаемых платформ. На втором
месте стоит Avare, за счет открытости и поддержке платформы Android, на которой
работает большое число взаимозаменяемого оборудования.

\newpage
\chapter*{Заключение}
\addcontentsline{toc}{chapter}{Заключение}

Целью написания данного дипломного проекта являлась разработка системы индикации
для транспондера VDL--4. Использование этой системы облегчит обустройство кабины
пилота самолета и рабочих мест наземных станций за счет облегчения технических
требований к оборудованию и устранения необходимости ставить строго определенное
совместимое оборудование. Помимо этого, технологически данная система индикации
является прочным высокотехнологичным фундаментом для дальнейшего развития а так
же для построения других, аналогичных систем.

В первом разделе данной работы рассматривались основные проблемы возникающие в
сфере авиационной связи и наблюдения, ставится акцент на важности необходимости
знать обстановку в воздушном пространстве в данный момент, в частности в
областях с густонаселенном воздушным трафиком.

Во втором разделе была описана концепция автономной сети между воздушными и
наземными объектами. Сеть, именуемая просто ``А--Сеть'', самоорганизующаяся: ей
не нужны центральные узлы связи, которые бы координировали её. Так же в разделе
были описаны требования ``А--Сети'' к оборудованию узлов.

В третьем разделе были разработаны требования которым должна соответствовать
система индикации, помимо основным требований были установлены требованию по
поддерживаемой аппаратуре.

% БАЗЗВОРДЫ ДЛЯ БОГА БАЗЗВОРДОВ
Четвертый раздел посвящен непосредственно разработке системы индикации.
Произведен выбор языка программирования, инфраструктура, сторонние библиотеки и
фреймворк. Приводится листинг исходного кода программы, иллюстрация
разработанного интерфейса.

Пятый раздел посвящен разработке инструкции по эксплуатации, в виду простоты
интерфейса инструкция по эксплуатации достаточно короткая.

Шестой раздел посвящен вопросам безопасности жизнедеятельности и охране труда
работника наземной станции при работе с электронным оборудованием - были
выявлены основные опасные факторы, пагубно влияющие на здоровье сотрудника, был
произведен расчет необходимой искусственной освещенности в помещении.

В седьмом разделе было произведено сравнение системы индикации с доступными
аналогами методом анализа иерархий. Система индикации была сравнена с двумя
другими приложениями, доступными на мобильных платформах -- Android и iOS.
Результаты анализа показали что система индикации разрабатываемая в данной
работе - наилучший вариант, в основном за счет своей платформонезависимости.
\newpage

%\chapter*{Список литературы}

\begin{thebibliography}{9}
\bibitem{theconcept}
  Концепция построения сети передачи информации для потребностей авионики на
  основе идеологии режима VDL-4 
\bibitem{ed108}
  EUROCAE ED-108A МИНИМАЛЬНЫЕ ТРЕБОВАНИЯ СТАНДАРТОВ К ФУНКЦИОНАЛЬНЫМ
  ХАРАКТЕРИСТИКАМ БОРТОВОГО ПРИЕМОПЕРЕДАТЧИКА РЕЖИМА VDL MODE 4 
\bibitem{electron}
  Electron JS \url{http://electron.atom.io/}
\bibitem{d3js}
  d3.js \url{https://d3js.org/}
\bibitem{ledare}
  IKEA LEDARE 702.667.65 \url{http://www.ikea.com/ru/ru/catalog/products/70266765/}
\bibitem{ledcoef}
  \url{http://www.malahit-irk.ru/index.php/2011-01-13-09-04-43/202-2011-07-07-12-57-50.html}
\bibitem{ledcoef2}
  \url{http://www.axiomasveta.com/info/koeffitsient_zapasa/}
\end{thebibliography}

\end{document}